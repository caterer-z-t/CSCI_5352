\documentclass[12pt]{article}

\usepackage[margin=1in]{geometry}
\usepackage{amsmath, amsthm}
\usepackage{amssymb}
\usepackage{graphicx}
\usepackage{times}
% \usepackage{hyperref}
\usepackage[colorlinks=true, linkcolor=black, citecolor=black, urlcolor=black]{hyperref}


\title{CSCI 5352: Homework 2}
\author{
    Zachary Caterer$^{1,2,3}$ \\
    \small $^1$Department of Computer Science, University of Colorado Boulder \\
    \small $^2$Department of Chemical and Biological Engineering, University of Colorado Boulder \\
    \small $^3$Interdisciplinary Quantitative Biology Program, University of Colorado Boulder
}
\date{\today}

\begin{document}

\maketitle

\section*{Problem 1}

\textit{\textbf{Need to fix, complete, and check this proof.}}\\
Let the total number of nodes in the network be \( n = n_A + n_B \). The closeness centrality of a node \( i \) is defined as:
\[
C_i = \frac{n-1}{\sum_{j=1}^{n} d(i,j)}
\]
where \( d(i) = \sum_{j=1}^{n} d(i,j) \) represents the sum of shortest path distances from node \( i \) to all other nodes.

We aim to show the relationship between the closeness centralities \( C_A \) and \( C_B \) of vertices \( A \) and \( B \). 

Define Distance Sums
The sum of distances from node \( A \) is:
\[
d(A) = \sum_{j=1}^{n} d(A,j) = \sum_{j \in A} d(A,j) + \sum_{j \in B} d(A,j)
\]
Similarly, for node \( B \):
\[
d(B) = \sum_{j=1}^{n} d(B,j) = \sum_{j \in B} d(B,j) + \sum_{j \in A} d(B,j)
\]

 Express Distance Relations
Define:
\[
S_A = \sum_{j \in A} d(A,j), \quad S_B = \sum_{j \in B} d(B,j)
\]
where \( S_A \) is the total sum of distances from \( A \) to all nodes in \( A \), and \( S_B \) is the sum of distances from \( B \) to all nodes in \( B \).

Since the two subnetworks are connected by a single edge, each node in \( B \) is one unit farther from \( A \) than from \( B \), and vice versa:
\[
\sum_{j \in B} d(A,j) = \sum_{j \in B} (d(B,j) + 1) = S_B + n_B
\]
\[
\sum_{j \in A} d(B,j) = \sum_{j \in A} (d(A,j) + 1) = S_A + n_A
\]

Compute \( d(A) \) and \( d(B) \)
\[
d(A) = S_A + (S_B + n_B) = S_A + S_B + n_B
\]
\[
d(B) = S_B + (S_A + n_A) = S_A + S_B + n_A
\]

Compute Closeness Centralities
\[
C_A = \frac{n-1}{d(A)} = \frac{n-1}{S_A + S_B + n_B}
\]
\[
C_B = \frac{n-1}{d(B)} = \frac{n-1}{S_A + S_B + n_A}
\]


Taking reciprocals:
\[
\frac{1}{C_A} = \frac{S_A + S_B + n_B}{n-1}
\]
\[
\frac{1}{C_B} = \frac{S_A + S_B + n_A}{n-1}
\]
Dividing both equations:
\[
\frac{\frac{1}{C_A}}{\frac{1}{C_B}} = \frac{S_A + S_B + n_B}{S_A + S_B + n_A} = \frac{n_B}{n_A}
\]
Rearranging:
\[
\frac{1}{C_A} n_A = \frac{1}{C_B} n_B
\]
Multiplying by \( \frac{1}{n} \):
\[
\frac{1}{C_A} n_A \frac{1}{n} = \frac{1}{C_B} n_B \frac{1}{n}
\]

Thus, we have proven the required relationship:
\[
\frac{1}{C_A} \frac{n_A}{n} = \frac{1}{C_B} \frac{n_B}{n}
\]

\section*{Problem 2}    

\textbf{\textit{Need to fix, complete, and check this section.}}\\
\subsection*{Part A}
\begin{enumerate}
    \item Selection Criteria:
    \begin{itemize}
        \item Randomly select two directed edges \( (u,v) \) and \( (x,y) \) from the network.
        \item Since it is a directed network, the order of the nodes in the edge matters. Thus, the two edges are distinct.
    \end{itemize}
    \item Possible output configurations:
    \begin{itemize}
        \item Option 1: \( (u, x)\)and \( (v, y) \)
        \item Option 2: \( (u, y)\)and \( (v, x) \)
    \end{itemize}
    \item Checkes needed on $G'$:
    \begin{itemize}
        \item Ensure no self loops.
        \item Ensure no multiple edges.
        \item Verify the $k^{in}$ and $k^{out}$ values of the nodes remain unchanged.
    \end{itemize}
\end{enumerate}

\subsection*{Part B}
\begin{enumerate}
    \item Selection Criteria:
    \begin{itemize}
        \item Randomly select two directed edges one from each partition, for simplicity sake lets say nodes \( u \) and \( v \) are from partition \( A \) and nodes \( x \) and \( y \) are from partition \( B \). So the two edges are \( (u,x) \) and \( (v,y) \).
    \end{itemize}
    \item Possible output configurations:
    \begin{itemize}
        \item The only option: \( (u, y)\) and \( (v, x) \)
    \end{itemize}
    \item Checkes needed on $G'$:   
    \begin{itemize}
        \item Ensure no self loops.
        \item Ensure no multiple edges.
        \item Verify the degrees of the nodes remain unchanged.
        \item Veriy the new edges maintain bipartite structure.
    \end{itemize}
\end{enumerate}

\section*{Problem 3}

\textbf{\textit{Need to fix, complete, and check this section.}}\\

\section*{Problem 4}
\section*{Problem 5}
\subsection*{Part A}
\subsection*{Part B}
\subsection*{Part C}

Extra credit. Might do if I have time.

\section*{Problem 6}

Extra credit. Might do if I have time. 

\section*{Problem 7}

I read the paper ``Predicting poverty and wealth from mobile phone metadata'' by Blumenstock et al. (2016).  The research question they wanted to answer was whether mobile phone metadata could be used to predict an individuals socioeconomic status. They used two main types of datasets from Rwanda's mobile network phone records of its users, and survey data from a subset of this population. The approach they took was to use the datasets and build a machine learning model to predict socioeconomic status. They extracted features from the mobile phone metadata, such as the number of calls and texts made, the number of unique contacts, etc. They then trained and tested their model using the survey data as ground truth.

The authors did a good job of leveraging a large and unique dataset to address their research question.  However, they could have improved their work by discussing potential biases in their dataset more thoroughly. For example, they could have explored how the exclusion of individuals without mobile phones might affect their findings. Additionally they had a limited geographic scope, since they only tested in Rwanda, which could limit the generalizability of their results.

One possible extension of this work could be to apply similar techniques to predict other social outcomes, such as health status or educational attainment. Additionally, future research could investigate the use of mobile phone metadata in different geographic regions to see if the findings are consistent across different contexts. I also think they should expland the geographicical scope of their study to see if their results are generalizable to other regions.


\end{document}